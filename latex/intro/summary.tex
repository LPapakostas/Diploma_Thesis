\pagestyle{plain}
\begin{center}
{\LARGE Περίληψη}\\[1cm]
\end{center}
Η παρούσα διπλωματική εργασία έχει ως αντικείμενο την εξέταση και την σύγκριση μεθόδων για την εξαγωγή συνιστωσών από γραμμικούς συνδυασμούς σημάτων, βασισμένο στο \en BSS \gr μοντέλο, που αφορούν το δυναμικό που παράγει η καρδιά.
\\
Παρουσιάζονται και υλοποιούνται με χρήση της \en Python \gr οι αλγόριθμοι \en FastICA \gr με κριτήριο την αρνητική εντροπία και π\en CA \gr με κριτήριο την περιοδικότητα των επιμέρους σημάτων.
\\
Τέλος, οι παραπάνω αλγόριθμοι εφαρμόζονται και συγκρίνονται σε τεχνητά περιοδικά σήματα αλλά στις βάσεις δεδομένων \en abdfecgdb \gr του \en Physionet \gr και \en DaISy \gr που περιέχουν ηλεκτροκαρδιογραφήματα από κυοφορούσες γυναίκες, χρησιμοποιώντας αντίστοιχα κριτήρια ως προς την εγγύτητα.

\blankpage

\pagestyle{plain}
\begin{center}
\en
{\LARGE Abstract}\\[1cm]
\end{center}
\en
The present diploma thesis examines and compares methods about extraction of components of linear mixed signals, based on BSS model, that refer to the voltage that heart produces.
\\
The algorithms that presented and implemented in Python are FastICA with negentropy criterion and \gr π\en CA with periodity of each component as criterion.
\\
Finally, the above methods are applied in synthesized periodic signals as well as abdfecgd database from Physionet and DaISy database, and compared with the use of proper criteria for their correctness.



