\gr 
\justifying
\section{Ορισμός} \label{sec:3.1}
Έστω ότι έχουμε ένα \en BSS \gr μοντέλο \en $\mathbf{X} = \mathbf{A} \mathbf{S}$ \gr όπου \en $\mathbf{X} = \begin{bmatrix}x_1 & x_2 & \ldots & x_n \end{bmatrix}^T$, $\mathbf{S} = \begin{bmatrix} s_1 & s_2 & \ldots & s_n \end{bmatrix}^T$, $\mathbf{A} \in \mathbb{R}^{n \times n}$ \gr και θέλουμε να βρούμε εκείνο τον γραμμικό συνδυασμό, \en $\mathbf{\hat{S}} = \mathbf{W} \mathbf{X}$, \gr ώστε οι παρατηρήσεις \en $\mathbf{\hat{S}}$ \gr να είναι στατιστικώς ανεξάρτητες. Αυτό είναι το πρόβλημα που επιλύει η \emph{ανάλυση ανεξαρτήτων συνιστωσών} \en (Independent Component Analysis) \cite{ica:6}, \cite{ica:7}.
\\ 
\gr Για να είμαστε σίγουροι ότι το μοντέλο \en ICA \gr μπορεί να εκτιμήσει τα σήματα πηγών κάνουμε τις ακόλουθες υποθέσεις:
\begin{itemize}
    \item Τα σήματα πηγών του πίνακα \en $\mathbf{S}$ \gr είναι \emph{στατιστικώς ανεξάρτητα}.
    \item Οι ανεξάρτητες συνιστώσες να ακολουθούν μη Γκαουσιανές κατανομές, καθώς στην περίπτωση Γκαουσιανών κατανομών οι ροπές ανώτερης τάξης είναι μηδενικές και κατά συνέπεια η εφαρμογή του αλγορίθμου \en ICA \gr είναι αδύνατη.
    \item Ο πίνακας μίξης \en $\mathbf{A}$ \gr είναι τετραγωνικός καθώς κατά την εκτίμηση του πίνακα \en $\mathbf{A}$, \gr μπορούμε να υπολογίσουμε τις ανεξάρτητες συνιστώσες από την αντιστροφή του, δηλαδή: \en
    \begin{align*}
        \mathbf{\hat{S}} = \mathbf{A}^{-1} \mathbf{X}
    \end{align*}
\end{itemize}
\gr Επίσης, για να μειώσουμε την πολυπλοκότητα του \en ICA, \gr θα κάνουμε κάποιες επιπλέον υποθέσεις:
\begin{itemize}
    \item Οι πηγές να έχουν μηδενικό μέσο όρο. Σε περίπτωση που είναι διάφορος του μηδενός, αφαιρούμε το μέσο όρο από τις παρατηρούμενες μεταβλητές \en $\mathbf{X}$ \gr καθώς και οι ανεξάρτητες συνιστώσες \en $\mathbf{\hat{S}}$ \gr θα έχουν και αυτές μηδενική μέση τιμή.
    \item Οι πηγές να έχουν πίνακα συσχέτισης ίσο με τον μοναδιαίο καθώς περιορίζει την εύρεση του πίνακα μίξης σε εύρεση ορθογώνιου πίνακα. Αν δεν είναι μοναδιαίος, μπορούμε να κάνουμε λεύκανση δεδομένων. 
\end{itemize}
Από τις πολλές μεθόδους που αναφέρονται στο \cite{ica:6}, θα ασχοληθούμε με τον αλγόριθμο \en FastICA \gr που έχει ως κριτήριο την μεγιστοποίηση της μη-γκαουσιανότητας, και συγκεκριμένα με την χρήση της αρνητικής εντροπίας.
% ta meionekthmata sto pararthma 
\section{Βοηθητικές Έννοιες} \label{sec:3.2}
\subsection{Μη Γκαουσιανότητα} \label{sec:3.2.1}
Για να αποφανθούμε για την μη γκαουσιανότητα, θα πρέπει να παραθέσουμε το ακόλουθο θεώρημα \cite{cltheorem:8}.
\begin{theorem} \label{th:3.1} [Κεντρικό Οριακό Θεώρημα]
Έστω η ακολουθία \en n \gr ανεξάρτητων τυχαίων μεταβλητών \en $X_1, X_2, \ldots, X_n$ που ακολουθούν την ίδια κατανομή, με μέση τιμή \en $E\{X_i\} = \mu$ \gr και διασπορά \en $ Var\{X_i\} = \sigma ^2 < \infty, \quad i = 1,2,\ldots,n$. \gr Τότε, καθώς το \en $n \rightarrow \infty $, η τυχαία μεταβλητή \en
\begin{align*}
    Z = \frac{1}{\sqrt{n \sigma^2}} \sum\limits_{i=1}^{n} \left ( X_i - \mu \right )
\end{align*}
\gr θα ακολουθεί ασυμπτωτικά την τυπική κανονική κατανομή $\mathcal{N}(0,1)$.
\end{theorem} 
Με άλλα λόγια, το άθροισμα δύο ανεξάρτητων τυχαίων μεταβλητών έχει συνήθως κατανομή πιο κοντά στην κανονική από οποιαδήποτε κατανομή των δύο αρχικών τυχαίων μεταβλητών.
\\ [0.5\baselineskip]
Ας υποθέσουμε τώρα ότι το διάνυσμα \en $x = \mathbf{A} s$ \gr είναι ένας συνδυασμός ανεξάρτητων συνιστωσών και για λόγους ευκολίας, υποθέτουμε ότι οι ανεξάρτητες συνιστώσες έχουν πανομοιότυπες κατανομές. Για να υπολογίσουμε μια από τις ανεξάρτητες συνιστώσες, θεωρούμε 
\begin{align*}
    y = w^T x = q^T s, \quad q = \mathbf{A}^T w
\end{align*}
έναν γραμμικό συνδυασμό των \en $x_i$ \gr με \en $w$ \gr το διάνυσμα που πρέπει να προσδιορίσουμε.
\\
Αν το \en $w$ \gr ήταν μια γραμμή του πίνακα \en $\mathbf{A}^{-1}$, \gr τότε ο γραμμικός συνδυασμός \en $y$ \gr θα ήταν ίσος με μια ανεξάρτητη συνιστώσα. Στην πράξη όμως, επειδή δεν γνωρίζουμε τον πίνακα \en $\mathbf{A}$, \gr μπορούμε να βρούμε μια εκτίμηση αυτού.
\\[0.5 \baselineskip]
Σύμφωνα με την σχέση \en $y = q^T s$, \gr  το \en $y$ \gr αποτελεί γραμμικό συνδυασμό των \en $s_i$, \gr με βάρη που δίνονται από τα \en $q_i$. \gr Σύμφωνα λοιπόν με το Κεντρικό Οριακό Θεώρημα, αφού το άθροισμα δύο ανεξάρτητων τυχαίων μεταβλητών είναι περισσότερο γκαουσιανό από τις αρχικές τυχαίες μεταβλητές, το \en $q^T s$ \gr θα είναι περισσότερο γκαουσιανό από οποιοδήποτε \en $s_i$ \gr και λιγότερο γκαουσιανό όταν στην πραγματικότητα γίνεται ίσο με ένα από τα \en $s_i$. \gr Σε αυτή την περίπτωση, μόνο ένα από τα \en $q_i$ \gr είναι μη μηδενικό.
\\[0.5 \baselineskip]
Επομένως, μπορούμε να θεωρήσουμε το \en $w$ \gr ως το διάνυσμα που μεγιστοποιεί την μη-γκαουσιανότητα του \en $w^T x$. \gr Ένα τέτοιο διάνυσμα θα ανταποκρινόταν αναγκαστικά σε ένα διάνυσμα \en $ q = \mathbf{A}^T w $ \gr το οποίο θα έχει μόνο μια μη μηδενική συνιστώσα. Αυτό σημαίνει ότι το \en $w^T x = q^T s$ \gr αποτελεί μια ανεξάρτητη συνιστώσα.
\subsection{Αρνητική Εντροπία} \label{sec:3.2.2}
Ένα μέτρο της μη προσαρμογής μιας κατανομής σε γκαουσιανή δίνεται από την αρνητική εντροπία (\en Negentropy). \gr Η αρνητική εντροπία βασίζεται στην ποσότητα της εντροπίας που προέρχεται από την θεωρία πληροφοριών \cite{entropy:9}.
\begin{definition} \label{def:3.1}
Η \emph{εντροπία} \en $\mathbf{H}$ \gr μίας τυχαίας διακριτής μεταβλητής \en $\mathbf{X}$ \gr ορίζεται ως : 
\begin{align*}
    \mathbf{H}(X) = -\sum\limits_{x \in X} p(x) \log p(x)
\end{align*}
\gr όπου τα \en $x$ \gr αποτελούν πιθανές τιμές της \en $X$. 
\end{definition}
\begin{definition} \label{def:3.2}
Η \emph{διαφορική εντροπία} \en $\mathbf{H}$ \gr μίας τυχαίας μεταβλητής \en $X$ \gr με συνάρτηση πυκνότητας πιθανότητας \en $f(x)$ \gr είναι:
\begin{align*}
    \mathbf{H}(X) = - \int f(x) \log f(x) dx
\end{align*}
\end{definition}
\gr Η εντροπία μιας τυχαίας μεταβλητής είναι συνυφασμένη με την πληροφορία που δίνει μια παρατήρηση μιας μεταβλητής. Όσο πιο απρόβλεπτη είναι μια μεταβλητή, τόσο μεγαλύτερη είναι η εντροπία της.
\\ [0.5 \baselineskip]
Ένα θεμελιώδες συμπέρασμα στην θεωρία πληροφορίας είναι ότι μια γκαουσιανή τυχαία μεταβλητή έχει την μεγαλύτερη εντροπία ανάμεσα σε όλες τις τυχαίες μεταβλητές που έχουν την ίδια διασπορά. Αυτό σημαίνει ότι η εντροπία μπορεί να χρησιμοποιηθεί ως μέτρο της μη γκαουσιανότητας. Στην πραγματικότητα, αυτό δείχνει ότι η γκαουσιανή κατανομή είναι πιο 'τυχαία' ή λιγότερο δομημένη από τις υπόλοιπες τυχαίες μεταβλητές.
\\ [0.5 \baselineskip]
Για να αποκτήσουμε ένα μέτρο της μη γκαουσιανότητας που να είναι ίσο με το μηδέν για γκαουσιανή τυχαία μεταβλητή και πάντα μη αρνητικό, με βάση την διαφορική εντροπία, ορίζουμε την \emph{αρνητική εντροπία}
\begin{definition} \label{def:3.3}
Η \emph{αρνητική εντροπία} \en $\mathbf{J}$ \gr ορίζεται ως: \en
\begin{align*}
    \mathbf{J}(X) = \mathbf{H}(X_{gauss}) - \mathbf{H}(X)
\end{align*}
\gr όπου \en $X_{gauss}$ \gr είναι μια γκαουσιανή τυχαία μεταβλητή με διασπορά ίση με την μεταβλητή \en $X$.
\end{definition}
\gr Με βάση τον Ορισμό \ref{def:3.3}, η αρνητική εντροπία είναι μηδενική όταν η τυχαία μεταβλητή \en $X$ \gr είναι γκαουσιανή και θετική στις υπόλοιπες περιπτώσεις. Επιπλέον, η αρνητική εντροπία παραμένει αμετάβλητη σε γραμμικούς αντιστρέψιμους μετασχηματισμούς, όπως ο \en PCA \gr και ο μετασχηματισμός λεύκανσης. Τέλος, το πλεονέκτημα της χρήσης της αρνητικής εντροπίας ως κριτήριο της μη γκαουσιανότητας είναι ότι η αρνητική εντροπία είναι καλά τεκμηριωμένη από την στατιστική θεωρία καθώς όσον αφορά τις στατιστικές ιδιότητες, αποτελεί τον βέλτιστο εκτιμητή της μη προσαρμογής σε γκαουσιανή.
\newpage
\noindent Το μειονέκτημα όσον αφορά την αρνητική εντροπία μιας τυχαίας μεταβλητής είναι ότι για τον ακριβή υπολογισμό της χρειάζεται την εκτίμηση της συνάρτησης πυκνότητας πιθανότητας της. Για την αποφυγή της παραπάνω διαδικασίας χρησιμοποιείται η ακόλουθη προσέγγιση \cite{ica:6}: \en
\begin{align} \label{eq:3.2.1}
    \mathbf{J}(X) \approx \left [ E\{ G(X)\} - E\{G(V)\} \right ]^2
\end{align}
\gr όπου \en $V$ \gr είναι μια γκαουσιανή τυχαία μεταβλητή με μηδενική μέση τιμή και μοναδιαία διασπορά και \en $G(.)$ \gr μια μη τετραγωνική συνάρτηση. Επίσης, υποθέτουμε ότι η τυχαία μεταβλητή \en $X$ \gr έχει και αυτή μηδενική μέση τιμή και μοναδιαία διασπορά και επιπλέον η κατανομή της \en $X$ \gr  πρέπει να είναι συμμετρική.
\\ [0.5 \baselineskip]
Το μόνο ζήτημα που προκύπτει με την \eqref{eq:3.2.1} είναι ότι με την κατάλληλη επιλογή της συνάρτησης \en $G$ \gr προκύπτουν καλύτερες προσεγγίσεις της \en $J$. \gr Συγκεκριμένα επιλέγονται συναρτήσεις οι οποίες δεν μεταβάλλονται πολύ γρήγορα, όπως: \en
\begin{subequations}
\begin{align}
&G(x) = \frac{1}{a} \log (\cosh (ax)) \quad 1 \leq a \leq 2 \label{eq:3.2.2a} \\
&G(x) = - \exp (- \frac{x^2}{2}) \label{eq:3.2.2b} \\
&G(x) = \frac{1}{4} x^4 \label{eq:3.2.2c}
\end{align}
\end{subequations} \gr
\section{Αλγόριθμος \en FastICA} \label{sec:3.3}
\gr Ο αλγόριθμος \en FastICA \gr \cite{ica:10} αποτελεί έναν \en fixed-point \gr αλγόριθμο, ο οποίος βρίσκει μια διεύθυνση, με άλλα λόγια ένα διάνυσμα \en $w$, \gr έτσι ώστε η προβολή του \en $w^T x$ \gr να μεγιστοποιεί την μη γκαουσιανότητα.
\\ [0.5 \baselineskip]
Με τον όρο \en fixed-point, \gr ένας αλγόριθμος υλοποιεί ένα μεγάλο μέρος των υπολογισμών σε ένα μόνο βήμα του. Αυτό έχει σαν αποτέλεσμα ότι τέτοιοι αλγόριθμοι, όπως και ο \en FastICA, \gr μπορούν να υλοποιηθούν παράλληλα, είναι υπολογιστικά απλοί και χρησιμοποιούν λίγη μνήμη.
\\ [0.5 \baselineskip]
Η μη γκαουσιανότητα μετριέται μέσω της προσέγγισης της αρνητικής εντροπίας \eqref{eq:3.2.1}. Επιπλέον, υπάρχει ο περιορισμός ότι η διασπορά του \en $w^T x$ \gr πρέπει να είναι μοναδιαία. Για λευκά δεδομένα, αυτό ισοδυναμεί με τον περιορισμό ότι το μέτρο του διανύσματος \en $w$ \gr να είναι ίσο με την μονάδα.
\\ [0.5 \baselineskip]
Ο υπολογισμός μιας ανεξάρτητης συνιστώσας μέσω του αλγορίθμου \en FastICA \gr είναι ο ακόλουθος, με την απόδειξη του να αναφέρετε στο \cite{ica:10}:
\begin{enumerate}
    \item Επιλογή ενός τυχαίου διανύσματος \en $w$ με μοναδιαίο μέτρο.
    \item Θέτουμε \en $ w \leftarrow E\left\{ x g(w^T x) \right\} - E\left\{ g^{'}(w^T x)  \right\} w$\gr
    \item Κανονικοποιούμε \en $w \leftarrow \frac{w}{\parallel w \parallel }$  \gr 
    \item Αν ο αλγόριθμος δεν συγκλίνει, τότε επιστρέφουμε στο βήμα 2.
\end{enumerate}
Οι συναρτήσεις \en $g$ \gr και \en $g^{'}$ \gr αποτελούν την πρώτη και την δεύτερη παράγωγο αντίστοιχα των συναρτήσεων \en $G$ \gr που ορίσαμε στις σχέσεις \en \eqref{eq:3.2.2a} - \eqref{eq:3.2.2c} \gr, δηλαδή: \en
\begin{subequations}
\begin{align}
    &g_{1}(x) = \tanh (ax), \quad g_{1}^{'}(x) = a \left[ 1-\tanh^{2}(x) \right] \quad 1 \leq a \leq 2 \label{eq:3.3.1a} \\
    &g_{2}(x) = x \exp ( -\frac{x^2}{2}), \quad g_{2}^{'} = \left( 1-x^2 \right) \exp ( -\frac{x^2}{2})  \label{eq:3.3.1b} \\
    &g_{3}(x) = x^3, \quad g_{3}^{'} = 3x^2 \label{eq:3.3.1c}
\end{align}
\end{subequations}
\gr Η σύγκλιση, σε αυτήν την περίπτωση, σημαίνει ότι οι παλιές και νέες τιμές του \en $w$ \gr δείχνουν προς την ίδια κατεύθυνση, δηλαδή το εσωτερικό τους γινόμενο είναι περίπου ίσο με την μονάδα. Αξίζει να σημειωθεί ότι δεν είναι ανάγκη το διάνυσμα \en $w$ \gr να συγκλίνει σε ένα μόνο σημείο, αφού το \en $w$ \gr και το \en $-w$ \gr έχουν την ίδια διεύθυνση.
\\ [0.5 \baselineskip]
Η παραπάνω διαδικασία δεν αποτελεί μια αξιόπιστη μέθοδο ως προς την εύρεση ανεξάρτητων συνιστωσών καθώς χρησιμοποιούμε πάρα πολλές αρχικές συνθήκες για κάθε μια συνιστώσα και υπάρχει περίπτωση δύο διαφορετικά διανύσματα να συγκλίνουν στο ίδιο μέγιστο. 
\\ [0.5 \baselineskip]
Κάνοντας ένα μετασχηματισμό λεύκανσης στα δεδομένα, παρατηρούμε ότι τα διανύσματα \en $w_i$ \gr είναι μεταξύ τους ορθογώνια. Με άλλα λόγια, η ανεξαρτησία των συνιστωσών προϋποθέτει ότι τα δεδομένα μας είναι ασυσχέτιστα που μεταφράζεται σε ορθογωνιότητα, κάνοντας \en data whitening. \gr Άρα, για να υπολογίσουμε κάποιες ανεξάρτητες συνιστώσες, πρέπει να τρέξουμε τον αλγόριθμο \en FastICA \gr αρκετές φορές και σε κάθε επανάληψη να κάνουμε τα διανύσματα \en $w_i$ \gr ορθογώνια μεταξύ τους. 
\\ [0.5 \baselineskip]
Με τις παρακάτω μεθόδους, επιτυγχάνετε η αποσυσχέτιση των ανεξαρτήτων συνιστωσών:
\begin{itemize}
    \item \textbf{\en Deflation \gr μέθοδος}
    \\ 
    Η μέθοδος αυτή βασίζεται στη ορθοκανονικοποίηση \en Gram-Schmidt \cite{gram-smidt:11} \gr που σημαίνει ότι υπολογίζουμε τις ανεξάρτητες συνιστώσες μία προς μία. Όταν έχουμε υπολογίσει \en $p$ \gr ανεξάρτητες συνιστώσες ή \en $p$ \gr διανύσματα \en $w_1, w_2, \ldots, w_p$, \gr εφαρμόζουμε τον αλγόριθμο \en FastICA \gr για τον υπολογισμό του \en $w_{p+1}$ \gr και μετά από κάθε επανάληψη, αφαιρούμε από το \en $w_{p+1}$ \gr τις προβολές των προηγούμενων \en $p$ \gr διανυσμάτων \en $ \left ( w_{p+1}^T w_{j} \right ) w_{j} , j = 1, 2, \ldots, p$ \gr και το κανονικοποιούμε. Δηλαδή: \en
    \begin{align} \label{eq:3.3.2}
        w_{p+1} \leftarrow w_{p+1} - \sum\limits_{j=1}^{p} \left( w_{p+1}^T w_j  \right)w_j
    \end{align}
    \gr και \en
    \begin{align} \label{eq:3.3.3}
        w_{p+1} \leftarrow \frac{w_{p+1}} { \parallel w_{p+1} \parallel }
    \end{align}
    \gr
    \newpage
    \item \textbf{Συμμετρική προσέγγιση}
    \\
    Τα διανύσματα \en $w_i$ \gr υπολογίζονται παράλληλα, σε αντίθεση με την προηγούμενη μέθοδο, κάνοντας παράλληλη εύρεση ανεξαρτήτων συνιστωσών. Η ορθογωνοποίηση του πίνακα των διανυσμάτων \en $\mathbf{W} = \begin{bmatrix}
    w_1 & w_2 & \ldots & w_n \end{bmatrix}^T$ \gr γίνεται με τον υπολογισμό της τετραγωνικής ρίζας πινάκων: \en
    \begin{align} \label{3.3.4}
        \mathbf{W} \leftarrow \left ( \mathbf{W} \mathbf{W}^T  \right ) ^{-\frac{1}{2}} \mathbf{W}
    \end{align}
    \gr όπου ο πίνακας \en $ \left ( \mathbf{W} \mathbf{W}^T  \right )^{-\frac{1}{2}} $ \gr λαμβάνεται, σύμφωνα με το Θεώρημα \ref{th:1.4}, ως \en $\mathbf{V} \mathbf{D}^{-\frac{1}{2}} \mathbf{V}^T$, \gr με \en $\mathbf{D}$ \gr τον διαγώνιο πίνακα των ιδιοτιμών και \en $\mathbf{V}$ τον πίνακα των ιδιοδιανυσμάτων.
    \\ 
    Μια πιο απλή εναλλακτική μέθοδος είναι επίσης η εξής διαδικασία:
    \begin{enumerate}
        \item Θέτουμε \en $\mathbf{W} \leftarrow \frac{\mathbf{W}}{\parallel \mathbf{W} \parallel}$  \gr
        \item Θέτουμε \en $\mathbf{W} \leftarrow \frac{3}{2} \mathbf{W} - \frac{1}{2} \mathbf{W} \mathbf{W}^T  \mathbf{W}$ \gr 
        \item Εάν ο πίνακας \en $\mathbf{W} \mathbf{W}^T $ \gr δεν είναι κοντά στον μοναδιαίο, τότε επιστροφή στο βήμα 2.
    \end{enumerate}
\end{itemize}
\section{Μεθοδολογία} \label{sec:3.4}
\justifying
Σε αυτήν την εργασία, θα υλοποιήσουμε τον αλγόριθμο \en FastICA \gr και με τις δύο προσεγγίσεις που αναφέραμε στο Κεφάλαιο
\ref{sec:3.3}. 
\\
Για την \en Deflation \gr προσέγγιση, ο αλγόριθμος που θα υλοποιήσουμε είναι ο εξής:
\begin{enumerate}
    \item Αφαιρούμε τον μέσο όρο από τα δεδομένα μας: \en $x \leftarrow x - m_x $ \gr
    \item Κάνουμε λεύκανση στα δεδομένα: \en $z = \left ( \mathbf{D}^{- \frac{1}{2}} \mathbf{V}^T \right)x $ \gr 
    \item Επιλέγουμε \en $m$ \gr ανεξάρτητες συνιστώσες για να εκτιμήσουμε.
    \item Θέτουμε τον μετρητή \en $p \leftarrow 1$ \gr
    \item Επιλέγουμε τυχαία ένα αρχικό διάνυσμα \en $w_p$ \gr με μοναδιαίο μέτρο.
    \item Θέτουμε \en $w_p \leftarrow E \left \{ z g( w_p^T z) \right \} - E \left \{ g^{'}(w_p^T z) \right \}w_p $, \gr όπου οι συναρτήσεις \en $g, g^{'} $ \gr ορίζονται από τις σχέσεις \en \eqref{eq:3.3.1a} - \eqref{eq:3.3.1c} \gr.
    \item Κάνουμε την παρακάτω ορθογωνοποίηση: \en
    \begin{align*}
        w_p \leftarrow w_p - \sum\limits_{j=1}^{p-1} \left (
        w_p^T w_j\right ) w_j
    \end{align*} \gr
    \item Κανονικοποιούμε \en $w_p \leftarrow \frac{w_p}{\parallel w_p \parallel} $ \gr
    \item Εάν το διάνυσμα \en $w_p$ \gr δεν έχει συγκλίνει, επιστρέφουμε στο βήμα 6.
    \item Αυξάνουμε το μετρητή \en $p \leftarrow p+1 $. \gr  
    Εάν \en $p \leq m$ , \gr επιστρέφουμε στο βήμα 5. 
\end{enumerate} \leavevmode
ενώ για στην Συμμετρική προσέγγιση:
\begin{enumerate}
    \item Αφαιρούμε τον μέσο όρο από τα δεδομένα μας: \en $x \leftarrow x - m_x $ \gr
    \item Κάνουμε λεύκανση στα δεδομένα: \en $z = \left ( \mathbf{D}^{- \frac{1}{2}} \mathbf{V}^T \right)x $ \gr 
    \item Επιλέγουμε \en $m$ \gr ανεξάρτητες συνιστώσες για να εκτιμήσουμε.
    \item Επιλέγουμε αρχικές συνθήκες για τα διανύσματα \en $w_i \gr , i = 1, 2, \ldots, m$ \gr με μόνη συνθήκη να έχουν όλα μοναδιαίο μέτρο.
    \item Κάνουμε συμμετρική ορθογωνοποίηση του πίνακα \en $\mathbf{W} = \begin{bmatrix} w_1 & \ldots & w_m \end{bmatrix}^T$ :
    \begin{align*}
        \mathbf{W} \leftarrow \left ( \mathbf{W} \mathbf{W}^T \right)^{-\frac{1}{2}} \mathbf{W}    
    \end{align*} \gr
    \item Για κάθε \en $i = 1, \ldots, m$, \gr 
    θέτουμε \en $w_i \leftarrow E \left \{ z g( w_i^T z) \right \} - E \left \{ g^{'}(w_i^T z) \right \}w_i $, \gr όπου οι συναρτήσεις \en $g, g^{'} $ \gr ορίζονται από τις σχέσεις \en \eqref{eq:3.3.1a} - \eqref{eq:3.3.1c} \gr.
    \item Κάνουμε ορθογωνοποίηση όπως το βήμα 4.
    \item Εάν ο πίνακας δεν συγκλίνει, επιστρέφουμε στο βήμα 6.
\end{enumerate}
Οι παραπάνω διαδικασίες υλοποιήθηκαν στους κώδικες \en fastica.py, deflational\_method.py \gr  και \en symmetric\_mehod.py \gr που βρίσκονται στο παράρτημα.
