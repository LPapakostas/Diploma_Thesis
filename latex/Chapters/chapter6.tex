Με τα αποτελέσματα των αλγορίθμων που παρουσιάσαμε και εκτελέσαμε στην παρούσα εργασία για τεχνητά και πραγματικά σήματα, μπορούμε να καταλήξουμε στα ακόλουθα συμπεράσματα:
\begin{enumerate}
    \item Η υπόθεση ότι όλα τα σήματα έχουν περιοδική δομή, ιδιαίτερα όταν ασχολούμαστε με βιοσήματα, είναι ' ισχυρότερη ' από την υπόθεση ότι όλα τα επιμέρους σήματα είναι ανεξάρτητα. Για αυτόν τον λόγο, ο αλγόριθμος π\en CA \gr υπερισχύει του αλγορίθμου \en FastICA \gr στην εξαγωγή συνιστωσών με περιοδική δομή.
    \item Ο αλγόριθμος π\en CA \gr είναι πιο αποδοτικός από άποψη χρόνου, καθώς απαιτείται μόνο ο υπολογισμός 3 πινάκων σε αντίθεση με τον επαναληπτικό αλγόριθμο σύγκλισης του \en FastICA. \gr
    \item Στον αλγόριθμο \en ICA \gr δεν είναι δυνατόν να προβλέψουμε την σειρά των εξαγώμενων συνιστωσών σε αντίθεση με τον π\en CA, \gr όπου οι εξαγώμενες συνιστώσες ταξινομούνται με βάση την περιοδικότητα τους.
    \item Οι ιδιοτιμές που υπολογίζονται από την σχέση \eqref{eq:4.1.10} αποτελούν ένα μέτρο της ύπαρξης θορύβου στις εξαγώμενες συνιστώσες και μπορούν να χρησιμοποιηθούν ως κατώφλι για την απόρριψη ασήμαντων συνιστωσών. Με άλλα λόγια, εάν θεωρήσουμε κάθε απεριοδικό σήμα πηγής ως θόρυβο, η μέθοδος π\en CA \gr μπορεί να ερμηνευτεί ως μετασχηματισμός που κατανέμει την διασπορά του θορύβου στις λιγότερο σημαντικές συνιστώσες. Με τον αλγόριθμο \en FastICA \gr δεν μπορεί να επιτευχθεί αυτό καθώς αναζητεί τις πιο ανεξάρτητες και όχι τις πιο περιοδικές (λιγότερο θορυβώδεις) συνιστώσες.
\end{enumerate}
\newpage
\noindent Τέλος, μερικές προτάσεις για μελλοντική έρευνα είναι:
\begin{itemize}
    \item η εύρεση μεθόδου συγχρονισμού των σημείων ενδιαφέροντος ώστε να μην υπάρχουν αλλαγές στην φάση των εξαγόμενων περιοδικών συνιστωσών και παρουσιάζονται ανεστραμμένες.
    \item η δημιουργία γραφικής διεπαφής (\en GUI) \gr για την εύρεση και την αναπαράσταση περιοδικών συνιστωσών.
    \item η ανάπτυξη μεθόδων εύρεσης βέλτιστου βήματος για την χρονική καθυστέρηση με σκοπό τον βέλτιστο συνδυασμό ακρίβειας και ταχύτητας.
    \item η δυνατότητα επέκτασης των μεθόδων για πιο περίπλοκα συστήματα \en BSS. \gr
    \item η εφαρμογή και σύγκριση των μεθόδων που παρουσιάστηκαν σε αυτή την εργασία και σε άλλα περιοδικά σήματα.
\end{itemize}